%-----------------------------------------------------------------------------------------------------------------------------------------------%
%	The MIT License (MIT)
%
%	Copyright (c) 2019 Jan Küster
%
%	Permission is hereby granted, free of charge, to any person obtaining a copy
%	of this software and associated documentation files (the "Software"), to deal
%	in the Software without restriction, including without limitation the rights
%	to use, copy, modify, merge, publish, distribute, sublicense, and/or sell
%	copies of the Software, and to permit persons to whom the Software is
%	furnished to do so, subject to the following conditions:
%	
%	THE SOFTWARE IS PROVIDED "AS IS", WITHOUT WARRANTY OF ANY KIND, EXPRESS OR
%	IMPLIED, INCLUDING BUT NOT LIMITED TO THE WARRANTIES OF MERCHANTABILITY,
%	FITNESS FOR A PARTICULAR PURPOSE AND NONINFRINGEMENT. IN NO EVENT SHALL THE
%	AUTHORS OR COPYRIGHT HOLDERS BE LIABLE FOR ANY CLAIM, DAMAGES OR OTHER
%	LIABILITY, WHETHER IN AN ACTION OF CONTRACT, TORT OR OTHERWISE, ARISING FROM,
%	OUT OF OR IN CONNECTION WITH THE SOFTWARE OR THE USE OR OTHER DEALINGS IN
%	THE SOFTWARE.
%	
%
%-----------------------------------------------------------------------------------------------------------------------------------------------%


%============================================================================%
%
%	DOCUMENT DEFINITION
%
%============================================================================%

%we use article class because we want to fully customize the page and don't use a cv template
\documentclass[10pt,A4]{article}


%----------------------------------------------------------------------------------------
%	ENCODING
%----------------------------------------------------------------------------------------

% we use utf8 since we want to build from any machine
\usepackage[utf8]{inputenc}		

%----------------------------------------------------------------------------------------
%	LOGIC
%----------------------------------------------------------------------------------------

% provides \isempty test
\usepackage{xstring, xifthen}

%----------------------------------------------------------------------------------------
%	FONT BASICS
%----------------------------------------------------------------------------------------

% some tex-live fonts - choose your own

%\usepackage[defaultsans]{droidsans}
%\usepackage[default]{comfortaa}
%\usepackage{cmbright}
\usepackage[default]{raleway}
%\usepackage{fetamont}
%\usepackage[default]{gillius}
%\usepackage[light,math]{iwona}
%\usepackage[thin]{roboto} 

% set font default
\renewcommand*\familydefault{\sfdefault} 	
\usepackage[T1]{fontenc}

% more font size definitions
\usepackage{moresize}

%----------------------------------------------------------------------------------------
%	FONT AWESOME ICONS
%---------------------------------------------------------------------------------------- 

% include the fontawesome icon set
\usepackage{fontawesome}

% use to vertically center content
% credits to: http://tex.stackexchange.com/questions/7219/how-to-vertically-center-two-images-next-to-each-other
\newcommand{\vcenteredinclude}[1]{\begingroup
\setbox0=\hbox{\includegraphics{#1}}%
\parbox{\wd0}{\box0}\endgroup}

% use to vertically center content
% credits to: http://tex.stackexchange.com/questions/7219/how-to-vertically-center-two-images-next-to-each-other
\newcommand*{\vcenteredhbox}[1]{\begingroup
\setbox0=\hbox{#1}\parbox{\wd0}{\box0}\endgroup}

% icon shortcut
\newcommand{\icon}[3] { 							
	\makebox(#2, #2){\textcolor{maincol}{\csname fa#1\endcsname}}
}	

% icon with text shortcut
\newcommand{\icontext}[4]{ 						
	\vcenteredhbox{\icon{#1}{#2}{#3}}  \hspace{2pt}  \parbox{0.9\mpwidth}{\textcolor{#4}{#3}}
}

% icon with website url
\newcommand{\iconhref}[5]{ 						
    \vcenteredhbox{\icon{#1}{#2}{#5}}  \hspace{2pt} \href{#4}{\textcolor{#5}{#3}}
}

% icon with email link
\newcommand{\iconemail}[5]{ 						
    \vcenteredhbox{\icon{#1}{#2}{#5}}  \hspace{2pt} \href{mailto:#4}{\textcolor{#5}{#3}}
}

%----------------------------------------------------------------------------------------
%	PAGE LAYOUT  DEFINITIONS
%----------------------------------------------------------------------------------------

% page outer frames (debug-only)
% \usepackage{showframe}		

% we use paracol to display breakable two columns
\usepackage{paracol}

% define page styles using geometry
\usepackage[a4paper]{geometry}

% remove all possible margins
\geometry{top=1cm, bottom=1cm, left=1cm, right=1cm}

\usepackage{fancyhdr}
\pagestyle{empty}

% space between header and content
% \setlength{\headheight}{0pt}

% indentation is zero
\setlength{\parindent}{0mm}

%----------------------------------------------------------------------------------------
%	TABLE /ARRAY DEFINITIONS
%---------------------------------------------------------------------------------------- 

% extended aligning of tabular cells
\usepackage{array}

% custom column right-align with fixed width
% use like p{size} but via x{size}
\newcolumntype{x}[1]{%
>{\raggedleft\hspace{0pt}}p{#1}}%


%----------------------------------------------------------------------------------------
%	GRAPHICS DEFINITIONS
%---------------------------------------------------------------------------------------- 

%for header image
\usepackage{graphicx}

% use this for floating figures
% \usepackage{wrapfig}
% \usepackage{float}
% \floatstyle{boxed} 
% \restylefloat{figure}

%for drawing graphics		
\usepackage{tikz}				
\usetikzlibrary{shapes, backgrounds,mindmap, trees}

%----------------------------------------------------------------------------------------
%	Color DEFINITIONS
%---------------------------------------------------------------------------------------- 
\usepackage{transparent}
\usepackage{color}

% primary color
\definecolor{maincol}{RGB}{ 0, 155, 255 }

% accent color, secondary
\definecolor{accentcol}{RGB}{ 250, 150, 10 }

% dark color
\definecolor{darkcol}{RGB}{ 70, 70, 70 }

% light color
\definecolor{lightcol}{RGB}{245,245,245}


% Package for links, must be the last package used
\usepackage[hidelinks]{hyperref}

% returns minipage width minus two times \fboxsep
% to keep padding included in width calculations
% can also be used for other boxes / environments
\newcommand{\mpwidth}{\linewidth-\fboxsep-\fboxsep}
	


%============================================================================%
%
%	CV COMMANDS
%
%============================================================================%

%----------------------------------------------------------------------------------------
%	 CV LIST
%----------------------------------------------------------------------------------------

% renders a standard latex list but abstracts away the environment definition (begin/end)
\newcommand{\cvlist}[1] {
	\begin{itemize}{#1}\end{itemize}
}

%----------------------------------------------------------------------------------------
%	 CV TEXT
%----------------------------------------------------------------------------------------

% base class to wrap any text based stuff here. Renders like a paragraph.
% Allows complex commands to be passed, too.
% param 1: *any
\newcommand{\cvtext}[1] {
	\begin{tabular*}{1\mpwidth}{p{0.98\mpwidth}}
		\parbox{1\mpwidth}{#1}
	\end{tabular*}
}

%----------------------------------------------------------------------------------------
%	CV SECTION
%----------------------------------------------------------------------------------------

% Renders a a CV section headline with a nice underline in main color.
% param 1: section title
\newcommand{\cvsection}[1] {
	\vspace{14pt}
	\cvtext{
		\textbf{\LARGE{\textcolor{darkcol}{\uppercase{#1}}}}\\[-4pt]
		\textcolor{maincol}{ \rule{0.1\textwidth}{2pt} } \\
	}
}

%----------------------------------------------------------------------------------------
%	META SKILL
%----------------------------------------------------------------------------------------

% Renders a progress-bar to indicate a certain skill in percent.
% param 1: name of the skill / tech / etc.
% param 2: level (for example in years)
% param 3: percent, values range from 0 to 1
\newcommand{\cvskill}[3] {
	\begin{tabular*}{1\mpwidth}{p{0.70\mpwidth}  r}
 		\textcolor{black}{\textbf{#1}} & \textcolor{maincol}{#2}\\
	\end{tabular*}%
	
	\hspace{4pt}
	\begin{tikzpicture}[scale=1,rounded corners=2pt,very thin]
		\fill [lightcol] (0,0) rectangle (1\mpwidth, 0.15);
		\fill [maincol] (0,0) rectangle (#3\mpwidth, 0.15);
  	\end{tikzpicture}%
}


%----------------------------------------------------------------------------------------
%	 CV EVENT
%----------------------------------------------------------------------------------------

% Renders a table and a paragraph (cvtext) wrapped in a parbox (to ensure minimum content
% is glued together when a pagebreak appears).
% Additional Information can be passed in text or list form (or other environments).
% the work you did
% param 1: time-frame i.e. Sep 14 - Jan 15 etc.
% param 2:	 event name (job position etc.)
% param 3: Customer, Employer, Industry
% param 4: Short description
% param 5: work done (optional)
% param 6: technologies include (optional)
% param 7: achievements (optional)
\newcommand{\cvevent}[7] {
	
	% we wrap this part in a parbox, so title and description are not separated on a pagebreak
	% if you need more control on page breaks, remove the parbox
	\parbox{\mpwidth}{
		\begin{tabular*}{1\mpwidth}{p{0.72\mpwidth}  r}
	 		\textcolor{black}{\textbf{#2}} & \colorbox{maincol}{\makebox[0.25\mpwidth]{\textcolor{white}{#1}}}\\
			\textcolor{maincol}{\footnotesize\textbf{#3}} & \\
		\end{tabular*}\\[4pt]
	
		\ifthenelse{\isempty{#4}}{}{
			\cvtext{#4}\\
		}
	}

	\ifthenelse{\isempty{#5}}{}{
		\vspace{9pt}
		\cvtext{\textbf{Activités :}}\\
		{#5}
	}

	\ifthenelse{\isempty{#6}}{}{
		\vspace{9pt}
		\cvtext{\textbf{Technologies et environnements :}}\\
		{#6}
	}

	\vspace{14pt}
}

%----------------------------------------------------------------------------------------
%	 CV META EVENT
%----------------------------------------------------------------------------------------

% Renders a CV event on the sidebar
% param 1: title
% param 2: subtitle (optional)
% param 3: customer, employer, etc,. (optional)
% param 4: info text (optional)
\newcommand{\cvmetaevent}[4] {
	\textcolor{maincol} {\cvtext{\textbf{\begin{flushleft}#1\end{flushleft}}}}

	\ifthenelse{\isempty{#2}}{}{
	\textcolor{darkcol} {\cvtext{\textbf{#2}} }
	}

	\ifthenelse{\isempty{#3}}{}{
		\cvtext{{ \textcolor{darkcol} {#3} }}\\
	}

	\cvtext{#4}\\[14pt]
}

%---------------------------------------------------------------------------------------
%	QR CODE
%----------------------------------------------------------------------------------------

% Renders a qrcode image (centered, relative to the parentwidth)
% param 1: percent width, from 0 to 1
\newcommand{\cvqrcode}[1] {
	\begin{center}
		\includegraphics[width={#1}\mpwidth]{qrcode}
	\end{center}
}


%============================================================================%
%
%
%
%	DOCUMENT CONTENT
%
%
%
%============================================================================%
\begin{document}
\columnratio{0.31}
\setlength{\columnsep}{2.2em}
\setlength{\columnseprule}{4pt}
\colseprulecolor{lightcol}
\begin{paracol}{2}
\begin{leftcolumn}
%---------------------------------------------------------------------------------------
%	META IMAGE
%----------------------------------------------------------------------------------------
%\includegraphics[width=\linewidth]{untitled.jpg}	%trimming relative to image size

%---------------------------------------------------------------------------------------
%	META SKILLS
%----------------------------------------------------------------------------------------
\cvsection{Compétences}

\cvskill{C++} {20+ ans} {1} \\[-4pt]

\cvskill{Qt 5} {10+ ans} {0.85} \\[-4pt]

\cvskill{Python} {10+ ans} {0.95} \\[-4pt]

\cvskill{Rust} {1+ an} {0.45} \\[-4pt]

\cvskill{Javascript/Typescript} {5+ ans} {0.75} \\[-4pt]

\cvskill{Html/CSS} {5+ ans} {0.66} \\[-4pt]

\cvskill{Docker} {5+ ans} {0.85} \\[-4pt]

\cvskill{Ansible} {2+ ans} {0.7} \\[-4pt]

\cvskill{Linux Embarqué} {8+ ans} {0.8} \\[-4pt]

\cvskill{Linux Administration} {15+ ans} {0.95} \\[-4pt]

\cvskill{Design Patterns} {15+ ans} {0.8} \\[-4pt]

\cvskill{UML} {15+ ans} {0.9} \\[-4pt]

\cvskill{Git} {8+ ans} {0.85} \\[-4pt]

\cvskill{Agilité} {10+ ans} {0.85} \\[-4pt]


\vfill\null


\cvsection{ME CONTACTER}
	
\icontext{MapMarker}{12}{11 TER Rue du Portereau, 44230 Saint-Sébastien-Sur-Loire}{black}\\[6pt]
\icontext{MobilePhone}{12}{+33 6 14 62 51 49}{black}\\[6pt]
\iconemail{Envelope}{12}{pbouamriou@me.com}{pbouamriou@me.com}{black}\\[6pt]

\vfill\null
\cvqrcode{0.5}

%---------------------------------------------------------------------------------------
%	EDUCATION
%----------------------------------------------------------------------------------------
\newpage


\cvsection{DIPLÔMES}

\cvmetaevent
{1995-1998}
{Ingénieur Télécom}
{Télécom Sud Paris (anciennement INT)}
{Spécialisation en Informatique parallèle et répartie}

\cvmetaevent
{1994-1995}
{Maitrise en Informatique}
{Université de Bretagne Occidentale}
{Mention Assez Bien}

\cvmetaevent
{1993-1994}
{Licence en Informatique}
{Université de Bretagne Occidentale}
{Mention Bien}

\cvmetaevent
{1991-1993}
{BTS en Informatique Industrielle}
{La Rochelle}
{}

\cvmetaevent
{1991}
{BAC Électrotechnique}
{Niort}
{Mention Assez Bien}

\vfill\null
\cvqrcode{0.5}

%---------------------------------------------------------------------------------------
%	LANGUES
%----------------------------------------------------------------------------------------

\newpage

\cvsection{Langues}

\cvskill{Français} {Natale} {1} \\[-4pt]

\cvskill{Anglais Parlé} {A2} {0.5} \\[-4pt]

\cvskill{Anglais Ecrit} {B1} {0.5} \\[-4pt]

\cvskill{Anglais Technique} {B1} {0.7} \\[-4pt]

\vfill\null

\mbox{} % hotfix to place qrcode on the bottom when there are not other elements
\vfill\null
\cvqrcode{0.5}

%---------------------------------------------------------------------------------------
%	REFERENCES
%----------------------------------------------------------------------------------------
\newpage
\cvsection{Références}

\cvmetaevent
{THALES - Thierry Nambot}
{\small Chef de Laboratoire (2008-2022)}
{
\iconemail{Envelope}{5}{\footnotesize thierry.nambot@thalesgroup.com}{thierry.nambot@thalesgroup.com}{black}\\
}
{}

\cvmetaevent
{THALES - Vincent Grégoire}
{\small Chef de Projet (2012-2020)}
{
\iconemail{Envelope}{5}{\footnotesize vincent.gregoire@thalesgroup.com}{vincent.gregoire@thalesgroup.com}{black}\\
}
{}

\cvmetaevent
{THALES - Nicolas Chevalier}
{\small Chef de Projet (2008/2012)}
{
\iconemail{Envelope}{5}{\footnotesize nicolas.chevalier@thalesgroup.com}{nicolas.chevalier@thalesgroup.com}{black}\\
}
{}

\mbox{} % hotfix to place qrcode on the bottom when there are not other elements

\vfill\null
\cvqrcode{0.5}

\end{leftcolumn}
\begin{rightcolumn}
%---------------------------------------------------------------------------------------
%	TITLE  HEADER
%----------------------------------------------------------------------------------------
\fcolorbox{white}{darkcol}{\begin{minipage}[c][3.5cm][c]{1\mpwidth}
	\begin {center}
		\HUGE{ \textbf{ \textcolor{white}{ \uppercase{Philippe BOUAMRIOU} } } } \\[-24pt]
		\textcolor{white}{ \rule{0.2\textwidth}{1.25pt} } \\[4pt]
		\large{ \textcolor{white} {Architecte Logiciel} }
	\end {center}
\end{minipage}} \\[14pt]
\vspace{-12pt}

%---------------------------------------------------------------------------------------
%	PROFILE
%----------------------------------------------------------------------------------------
\vfill\null
\cvsection{PROFIL}

\cvtext{Architecte Logiciel IHM Embarquée / MiddleWare / Web (Front + Back Embarqué).\\

J'ai conçu de nombreux logiciels/frameworks et j'accompagne les équipes durant le processus complet de développement à travers des revues de code, la création d'outils, la mise en place d'environnements de développement et en partageant continuellement mes connaissances acquises par de la veille technologique.\\

Je spécifie les interfaces, effectue les choix technologiques (protocoles, langages, frameworks, réutilisation, ...), définis les architectures logicielles et mets en place les stratégies de tests.
}

%---------------------------------------------------------------------------------------
%	WORK EXPERIENCE
%----------------------------------------------------------------------------------------
\vfill\null
\cvsection{Expériences Professionnelles}

\cvevent
	{Nov 21 - Maint.}
	{Architecte Logiciel FullStack Embarqué}
	{Thales}
	{Dans le cadre de la réalisation d'une nouvelle génération de modem radio Troposphère, j'ai participé à la mise en oeuvre d'une IHM Web pour configurer et superviser l'état du modem.}
	{\cvlist{
		\item Spécification / Architecture / Mise en place de l'environnement / Développement / Formation
	}}
	{\cvlist {
		\item C++17 / CMake pour le Back, HTML5 / SCSS / Typescript pour le Front
		\item Docker + Ansible pour le déploiement de l'environnement de Dev./Test en 1 click
		\item Python / Cypress pour les tests E2E (End To End)
		\item Linux Debian 11 (cible et natif)
	}}
	{\cvlist{
		\item Démonstration des différentes versions à chaque fin de Sprint (Méthode SCRUM)
		\item Environnement de test/dev réutilisé par d'autres équipes du Laboratoire
		\item Framework C++17 réutilisé par d'autres équipes du Laboratoire
	}}

\cvevent
	{Dec 16 - Nov 21}
	{Architecte Logiciel IHM Embarqué et WEB}
	{Thales}
	{Dans le cadre de la restauration des équipements radio VHF/UHF de l'armée française (CONTACT), réalisation de l'ensemble des IHM (IHM locale et déport WEB) de ces équipements.}
	{\cvlist{
		\item Spécification / Architecture / Mise en place de l'environnement / Développement / Formation
		\item Définition de la stratégie de Test
		\item Intégration sur cible
	}}
	{\cvlist {
		\item C++11 /Qt5 / QMake pour le développement de l'IHM embarquée
		\item Typescript / SCSS / HTML5 / Webpack pour le développement de l'IHM Web (Front)
		\item AppArmor pour la sécurisation des composants de l'IHM
		\item Docker / Python / Cypress pour les tests
		\item Processeur x86-64 / Linux Debian 10 (natif)
		\item Processeur ARM Freescale iMX6 Duo / Linux Elinos 6.2 (cible)
	}}
	{\cvlist{
		\item Livraison des deux premières versions système
	}}
	

\cvevent
	{Juil 16 - Dec 16}
	{Architecte Logiciel IHM Web}
	{Thales}
	{Dans le but d'enrichir l'expérience utilisateur du poste radio militaire PR4G 9215, réalisation d'une IHM Web embarquée dans celui-ci pour l'exploiter à distance sur un PC, une tablette, ou un smartphone.}
	{\cvlist{
		\item Architecture / Mise en place de l'environnement / Développement
	}}
	{\cvlist {
		\item Typescript / SCSS / HTML5 / Webpack pour le développement de l'IHM Web (Front)
		\item C++11 / Qt5 / Websockets pour l'adaptation de l'IHM locale
		\item Processeur ARM Microship SAMA5D4 (Cortex A5) / Linux Buildroot (cible)
	}}
	{\cvlist{
		\item Livraison du produit chez le premier client
	}}

\cvevent
	{Juin 16 - Dec 16}
	{Architecte Logiciel Middleware}
	{Thales}
	{Dans le cadre de la rénovation des équipements radio militaires VHF, réalisation d'un applicatif intégré dans l'équipement qui permet d'offrir des services de diffusion de positions GPS et d'envoi de messages courts.}
	{\cvlist{
		\item Architecture / Mise en place de l'environnement / Développement
	}}
	{\cvlist {
		\item C++11 / CMake pour le développement
		\item Linux / Windows
	}}
	{\cvlist{
		\item Intégration de l'applicatif dans l'équipement radio et livré au client (première version système)
	}}

\cvevent
	{Dec 12 - Juin 16}
	{Architecte Logiciel IHM Embarquée}
	{Thales}
	{Dans le cadre de l'évolution de l'IHM du poste radio militaire VHF PR4G, amélioration de l'expérience utilisateur en utilisant un écran LCD tactile en remplacement de l'écran 7 segments qui offrait une expérience datée.}
	{\cvlist{
		\item Spécification / Architecture / Mise en place de l'environnement / Développement
		\item Intégration sur cible
	}}
	{\cvlist {
		\item C++11 / Qt5 / QMake pour le développement de l'IHM
		\item Processeur ARM Microship SAMA5D4 (Cortex A5) / Linux Buildroot (cible) / Kernel Linux 3.10 
		\item Linux Debian 9 (natif)
		\item Python / Scripting / OpenSSL pour le logiciel de programmation des équipements
	}}
	{\cvlist{
		\item Equipements livré à plusieurs clients
	}}

\cvevent
	{Sep 08 - Dec 12}
	{Architecte Logiciel Middleware}
	{Thales}
	{Dans le cadre du développement d'un socle de messagerie militaire utilisant plusieurs média, soutien de l'ensemble des médias de communication (HF, VHF, LAN, RTC).}
	{\cvlist{
		\item Spécification / Architecture / Développement / Formation
	}}
	{\cvlist {
		\item C++03 / Boost / SCons pour le développement des composants
		\item Soap / XML / XSD pour la communication avec le socle de messagerie
		\item Windows XP
	}}
	{\cvlist{
		\item Socle de messagerie intégré dans différents SIC et livré à l'armée française
	}}

\cvevent
	{Sep 05 - Sep 08}
	{Référent Technique Logiciel Middleware}
	{SII / Thales}
	{Dans le cadre du développement d'un socle de messagerie militaire utilisant plusieurs médias, réalisation du média HF.}
	{\cvlist{
		\item Spécification / Architecture / Mise en place de l'environnement / Développement / Formation
	}}
	{\cvlist {
		\item C++03 / Boost / SCons pour le développement du plugin HF et du Framework de communication
		\item UML / IBM Rose / IBM Rhapsody pour la conception et génération de code (classes et automates)
		\item Java pour le simulateur de réseau radio HF
		\item Windows NT
	}}
	{\cvlist{
		\item Socle de messagerie intégré dans différents SIC et livré à l'armée française
	}}
	
\cvevent
	{Mars 03 - Sep 05}
	{Référent Technique Logiciel Middleware}
	{SII / Thales}
	{Mise à jour d'un logiciel de messagerie multi-média (HF/VHF/RTG/LAN) pour l'armée française.}
	{\cvlist{
		\item Spécification / Architecture / Développement / Tests
	}}
	{\cvlist {
		\item C++ pour le développement
		\item Windows NT
	}}
	{\cvlist{
		\item Le logiciel de messagerie a été intégré au projet CARTHAGE et livré à l'armée française
	}}

\cvevent
	{Nov 01 - Mars 03}
	{Référent Technique Logiciel Embarqué}
	{SII / Thales}
	{Dans le cadre de l'évolution du poste radio militaire VHF PR4G, ajout d'un coupleur IP permettant d'exploiter l'équipement radio à travers les protocoles IP (SNMP, UDP, TCP/IP)}
	{\cvlist{
		\item Conception / Mise en place environnement / Développement
	}}
	{\cvlist {
		\item C++ / Make pour le développement
		\item Unix - Sun Solaris (natif) / PSOS (cible)
	}}
	{\cvlist{
		\item Depuis 2003, l'armée française utilise le poste radio PR4G à travers les protocoles IP pour son administration et pour du transfert de données.
	}}

\cvevent
	{Sep 99 - Nov 01}
	{Référent Technique Logiciel IHM}
	{SII / Thales Training And Simulation}
	{Dans le cadre du Projet Helisim de simulation d'hélicoptère Tigre (Eurocopter), réalisation du logiciel du poste d'instruction.}
	{\cvlist{
		\item Spécification / Conception / Mise en place de l'environnement / Développement / Recettes
	}}
	{\cvlist {
		\item C++ / MFC pour le développement IHM et Python pour le traitement des données métier
		\item Windows NT (PC du poste instructeur) / VxWorks (Simulation et traitements métiers)
	}}
	{\cvlist{
		\item Le simulateur d'hélicoptère Tigre est parfaitement opérationnel au centre de formation d'Hélisim.
	}}
	
\cvevent
	{Mai 98 - Sep 99}
	{Développer IHM}
	{SII / Thales Training And Simulation}
	{Dans le cadre des ventes de simulateurs d'Airbus A320, A330 et A340 à différentes compagnie aériennes, adaptation du logiciel d'instruction par rapport aux besoins spécifiés par les clients.}
	{\cvlist{
		\item Spécification / Développement / Recettes
	}}
	{\cvlist {
		\item C / SGMLS pour le développement de l'IHM
		\item Unix Irix (natif et cible) 
	}}
	{\cvlist{
		\item Le simulateur d'hélicoptère Tigre est parfaitement opérationnel au centre de formation d'Hélisim.
	}}

% hotfixes to create fake-space to ensure the whole height is used
\mbox{}
\vfill

\end{rightcolumn}
\end{paracol}
\end{document}

